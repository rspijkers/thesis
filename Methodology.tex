\subsection{Introduction}
    This is a quick write-up of the methods, techniques, and settings used for the analysis. It aims to give a succinct yet complete description, specifically including details that may differ from the run 2 analysis.
    
    The goal of the analysis is to determine the correlation functions of cascade pairs, split into species ($\Xi$, $\Omega$) and sign difference (same-sign SS, opposite-sign OS). The correlation functions are presented as a function of $\Delta\varphi$ and/or $\Delta y$. 

    % ADD RELEVENT PIECES/LINKS TO THE CODE IN CERTAIN PLACES

\subsection{Selection criteria}
    \begin{table}[h]
        \centering
        \begin{tabular}{|c|c|}
            \hline
            \multicolumn{2}{|c|}{Cascade selection} \\
            \hline
            $|\eta_\text{tracks}|$ & $< 0.8$ \\
            \hline
            Number of TPC crossed rows & $> 50$ \\
            \hline
            TPC d$E$/d$x$ & $< 5\sigma$ \\
            \hline
            DCA V0 daughters & $< 0.5$ \\
            \hline
            DCA cascade daughters & $< 0.25$ \\
            \hline
            DCA pos V0 daughter to PV & $> 0.05$ \\
            \hline
            DCA neg V0 daughter to PV & $> 0.05$ \\
            \hline
            DCA V0 to PV & $> 0.03$ \\
            \hline
            DCA bach to PV & $> 0.04$ \\
            \hline
            V0 $\cos(\text{PA})$ & $> 0.9876$ \\
            \hline
            Cascade $\cos(\text{PA})$ & $> 0.9947$ \\
            \hline
            V0 radius & $> 0.55$ \\
            \hline
            Cascade radius & $> 1.01$ \\
            \hline
            V0 mass window & $< 11.6$ MeV/$c^2$\\
            \hline
            
        \end{tabular}
        \caption{Cascade selection criteria}
        \label{tab:cascsel}
    \end{table}

    % \begin{figure}[ht]
    %     \centering
    %     \begin{subfigure}[t]{.49\textwidth}
    %         \centering
    %         \includegraphics[width=\textwidth]{figures/EffXiMin.pdf}
    %         \caption{$\Xi^-$ efficiency for $|y| < 0.5$}
    %         \label{fig:Ximass06}
    %     \end{subfigure}
    %     \hfill
    %     \begin{subfigure}[t]{.49\textwidth}
    %         \centering
    %         \includegraphics[width=\textwidth]{figures/EffXiPlus.pdf}
    %         \caption{$\Xi^+$ efficiency for $|y| < 0.5$}
    %         \label{fig:Ximass10}
    %     \end{subfigure}

    %     \begin{subfigure}[t]{.49\textwidth}
    %         \centering
    %         \includegraphics[width=\textwidth]{figures/EffOmegaMin.pdf}
    %         \caption{$\Omega^-$ efficiency for $|y| < 0.5$}
    %         \label{fig:Ximass06}
    %     \end{subfigure}
    %     \hfill
    %     \begin{subfigure}[t]{.49\textwidth}
    %         \centering
    %         \includegraphics[width=\textwidth]{figures/EffOmegaPlus.pdf}
    %         \caption{$\Omega^+$ efficiency for $|y| < 0.5$}
    %         \label{fig:Ximass10}
    %     \end{subfigure}
    % \end{figure}

\subsection{Corrections}
    Corrections may be split into two different categories: single-particle corrections and pair corrections. The former consists mainly of a cascade reconstruction efficiency correction, while the latter is a mixed events correction that takes the ``two-particle efficiency" due to the detector acceptance into account
\subsubsection{Efficiency}
\subsubsection{Mixed events}

\subsection{MC closure test}
This section describes the methodology used for calculating the correlations at the MC truth level. It is very similar 